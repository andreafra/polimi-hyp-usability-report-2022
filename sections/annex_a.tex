% !TeX root = ../report.tex

\section{Annex A: Inspection}

% Col types:
% p - paragrafo ({3cm} specifica la larghezza)
% c - center (allinea al centro)
% X - espande per riempire la larghezza 
% (\linewidth = riempi la riga in orizzontale)
%     fa parte del pkg 'tabularx'
%
% '&' separa le colonne => nel nostro caso ne servono 3
% c1 & c2 & c3 & c4
% \\ <- linebreak/a capo
% \midrule <- disegna una riga orizzontale
%
In this sections there are the results of each inspector.

\begin{small}

\subsection{Evaluator: Alessio}
\begin{tabularx}{\linewidth}{c c X p{3cm}}
    % \toprule
    \textbf{Heuristic} & \textbf{Score} & \textbf{Comment} & \textbf{Page Url}
    \\ \midrule
    \endfirsthead
    \toprule
    \textbf{Heuristic} & \textbf{Score} & \textbf{Comment} & \textbf{Page Url}
    \\ \midrule
    \endhead
    \midrule
    \footnotesize [Continues on next page]
    \endfoot
    \bottomrule
    \endlastfoot
    % body
    H1 & 1 & Bread crumps are poorly used,and sometimes they show an incorrect path. Furthermore the search bar does not show any visual representation in the progression of searching & Homepage\\ \midrule
    H2 & 2 & Labels for section are brief but
    descriptive, easy to understand
    for the user (e.g. arriving and
    getting around, fly to Milan)
    Visual representations for specific
    sections are quite effective (e.g.
    showing pictures of art exhibitions
    as thumbnail for the art section in
    home, picture of a student for
    the study section in home), but
    some of them make no sense
    (e.g picture of a random
    building for the covid page) & Homepage, FAQ covid, Travel-info \\ \midrule
    H3 & 1 & Many pages have at the top
    some anchors leading to different
    sections of the same page,
    but users are not able to return to
    the top after clicking on any of the
    anchors, they have to scroll
    all the way back up
    Links leading to external sites
    don't have a warning saying that the
    user is leaving the site, nor it gets
    opened in a different tab
    Finally, due to their bad usage, bread
    crumps are not helpful in undoing the
    navigation (e.g. going through the
    navigation path backwards), so the user
    must rely on going back home or the
    "back" button of the browser & FAQ covid, Rents, Universities, 2-days Milano\\ \midrule
    H4 & 2 & The "+" most of the time works as intended,
    expanding the section, but
    occasionaly leads to a new page
    (e.g. in home). Furthermore sometimes
    the "-" is used in order to compress,
    while other times its role is replaced by
    an "x" & Castello Sforzesco, Homepage, events-month, miart\\ \midrule
    H5 & 2 & Subscribing
    to the newsletter with a
    missing @ in the email will result
    in an error message with suggestion,
    but if the same error is done in the student
    registration form no error prevention
    message appears
    Furthermore, some old links lead to
    non-existent pages, should be deleted & Study \\ \midrule
    H6 & 3 & In the search form for the venues,
    restaurants and hotels suggestions
    are present for type of venue
    and location & Venues, restaurants, hotels\\ \midrule
    H7 & 2 & Landmarks and shortcuts through the
    top menu ease a lot the navigation,
    though some options that are present
    in the footer are missing in the header & \\ \midrule
    H8 & 2 & Overall good, but some information
    overload in the menu and in the
    anchors mentioned before
    Also transitions and animations
    are sometimes present (e.g. carousel
    scrolling, sections expansion
    through the "+", reaching an anchor)
    and other times
    missing (e.g. dropdown of top menu) & \\ \midrule
    H9 & 2 & An incorrect email in the newsletter
    subscription will show a message
    that does not help in the
    error resolution,
    but incorrect data in the student
    form will suggest how to solve the
    problems.
    On the other hand, trying to access a
    non-existent page will result
    in a nice error message & \\ \midrule
    H10 & 3 & Privacy policies are clear & Privacy policy
\end{tabularx}
    
\begin{tabularx}{\linewidth}{l c c X p{3cm}}
\toprule
\textbf{Category} & \textbf{Heuristic} & \textbf{Score} & \textbf{Comment} & \textbf{Page Url} \\
\midrule
\endfirsthead
\toprule
\textbf{Category} & \textbf{Heuristic} & \textbf{Score} & \textbf{Comment} & \textbf{Page Url} \\
\midrule
\endhead
\midrule
\footnotesize [Continues on next page]
\endfoot
\bottomrule
\endlastfoot

\multirow{5}{*}{\textbf{Navigation}}   
    & MN1 & 3 & Similar pages offer the same
    type of interactions (e.g. itineries
    are all structured with a list of events,
    all venues show similar info such as
    price, opening time, location...) & \\ \cmidrule{2-5} 
    & MN2 & 3 & The top menu is always visible,
    letting users to switch between
    groups whenever they want &\\ \cmidrule{2-5} 
    & MN3 & 2 & As said before with anchors, going
    backwards to some parts of the same topic
    requires the user to scroll all the way back up
    (e.g. rent) &\\ \cmidrule{2-5} 
    & MN4 & 2 & In rents if one clicks on "read more"
    of a section, the bread crumps does not show
    the general page for rents in order to go back.
    For venues, an user can easily navigate to
    other venues by going back to the "venues"
    form page showed in the bread crump.
    Additionally a section dedicated
    to nearby and similar venues (you may also like)
    is very helpful.
    A similar section is present in
    events' pages for upcoming events & Rents, Rents-tenancy, Castello Sforzesco, miart, Venues\\ \cmidrule{2-5} 
    & MN5 & 2 & The only useful landmark is the one
    that leads to the home page,
    more landmarks could be added &\\ \midrule
\textbf{Content}                       & MC1 & 1 & A lot of information overload in the menu
and in some page anchors: the main
features should be maintained on the top
menu while putting the others in the footer.
Even worse, another criticity is the
absence of a link leading to the venues
search page (a very important
interaction page) in the top menu, while
the same link is present in the footer only &  \\ \midrule
\multirow{5}{*}{\textbf{Presentation}} & MP1 & 3 & Contrast of colors with the text is well
done (mostly black on white, if text
on an image, the text is wrapped inside
a box) and the
text-size is appropriate & \\ \cmidrule{2-5} 
    & MP2 & 3 & Icons are sometimes inconsistent
    ("-" replaced by "x"),
    but overall a lot of icons and
    brief descriptive labels are used
    that are pretty clear and coherent
    in their functionality & \\ \cmidrule{2-5} 
    & MP3 & 2 & There are some inconsistencies
    in the labels ("read more", "find out more",
    "see how", "discover how"...),
    that are functionally the same
    but worded differently & Rents\\ \cmidrule{2-5} 
    & MP4 & 2 & Some horizontal lines could be more
    marked to separate some parts within
    the same page (e.g. in the list of events
    of the month) &\\ \cmidrule{2-5} 
    & MP5 & 3 & &
\end{tabularx}

\pagebreak

\subsection{Evaluator: Andrea}
\begin{tabularx}{\linewidth}{c c X p{3cm}}
    % \toprule
    \textbf{Heuristic} & \textbf{Score} & \textbf{Comment} & \textbf{Page Url}
    \\ \midrule
    \endfirsthead
    \toprule
    \textbf{Heuristic} & \textbf{Score} & \textbf{Comment} & \textbf{Page Url}
    \\ \midrule
    \endhead
    \midrule
    \footnotesize [Continues on next page]
    \endfoot
    \bottomrule
    \endlastfoot
    % body
    H1 & 2 & The website is inconsistent in establishing the application status: most pages about a topic have breadcrumbs, but some pages with groups of topics do not. It's also not clear whether the website is loading search results, as the loading symbol is displayed at the bottom of the page with poor visibility. & Homepage, Itineraries, "La Rinascente" \\ \midrule
    H2 & 2 & Most of the website adopts words that a visitor to Milan could use, albeit some are misleading or change meaning between the English and Italian website (``Book now'' has a different meaning than ``Scopri come prenotare''). Pages such \emph{Arriving and Getting Around} follow a logical order in presenting their content (e.g.: ``Arriving`` section is before ``Getting Around''). & Homepage, Arriving and Getting Around \\ \midrule
    H3 & 2 & Most of the website doesn't let user perform actions beside navigating topics. It's easy to input queries that lead to no results in the Restaurant and Hotels sections of the website, and changing the query  resets the progress. & Hotels, Restaurants\\ \midrule
    H4 & 2 & While the style of most visual elements is consistent, links have ofter different styles (underlined, with an arrow at the end, in a box) whose use feels arbitrary, especially regarding external links. Both hotels and restaurant website have a completely different visual style. & Universities, ``Castello Sforzesco''\\ \midrule
    H5 & 1 & The form to subscribe to the newsletter only checks for email correctness after clicking the button. It's possible to get a ``You've been banned'' message on the Hotels and Restaurant section, which doesn't really work (refreshing the pages solves it), but prevents using the website for that session. & Homepage, Restaurants, Hotels \\ \midrule
    H6 & 3 & The website doesn't have any option to choose at all, and in forms (mostly on the Hotels and Restaurant sections) all possible options are listed below the input field. & Restaurants, Hotels  \\ \midrule
    H7 & 2 & The website doesn't have many accelerators: footer links, althought not as easily accessible, allow experienced users to quickly find relevant informations. The search button is useful and always present but it's quite slow. & N/A \\ \midrule
    H8 & 3 & The website has a modern and clean design, with good contrast. The only issue is the presentation of links which often are listed inline. & (e.g.) Arriving and Getting Around \\ \midrule
    H9 & 1 & While 404 pages on the main website presents links to get back to the website, hotels and restaurants sections have an unrecoverable ``You've been banned'' message. On the main website, the registration form for students explains clearly which fields have issues. & Hotels, Restaurants, Student registration \\ \midrule
    H10 & N/A & N/A & N/A
\end{tabularx}
 
\begin{tabularx}{\linewidth}{l c c X p{3cm}}
\toprule
\textbf{Category} & \textbf{Heuristic} & \textbf{Score} & \textbf{Comment} & \textbf{Page Url} \\
\midrule
\endfirsthead
\toprule
\textbf{Category} & \textbf{Heuristic} & \textbf{Score} & \textbf{Comment} & \textbf{Page Url} \\
\midrule
\endhead
\midrule
\footnotesize [Continues on next page]
\endfoot
\bottomrule
\endlastfoot

\multirow{5}{*}{\textbf{Navigation}}   
    & MN1 & 3 & Pages that show a group of topics have a distinct grid of boxes that easily differentiate the purpose of those pages with respect to the one about a topic. Pages about a topic are usually made by the same set of elements (carousels, panels, expandable text boxes). & (e.g.) Itineraries, ``La Rinascente'', ``Castello Sforzesco'' \\ \cmidrule{2-5} 
    & MN2 & 2 & While it's easy to navigate from a group to its single items, it's not always possible to navigate back to the group, since the only way to do it is through breadcrumbs. & (e.g.) ``La Rinascente'', ``Castello Sforzesco'' \\ \cmidrule{2-5} 
    & MN3 & 3 & Sections are well differentiated, and the panels accessible through the buttons (Map, Video, Info and Services, Contacts) are fairly intuitive. Longer pages often have a summary of topics at the top of page, which unfortunately isn't a sticky element. & (e.g.) ``La Rinascente'', Rents \\ \cmidrule{2-5} 
    & MN4 & 2 & It's easy to navigate to related topic through the carousels at the bottom of a topic page, but it's not possible to navigate back since the website doesn't hold the user navigation history. & (e.g.) ``La Rinascente'' \\ \cmidrule{2-5} 
    & MN5 & 2 & The user login button is not very useful since it doesn't have any useful feature associated with it. Other elements in the header are relatively useful. On a desktop computer it would have been better to have at list the main topics of the menu listed in the header. Hotels and Restaurants have useful landmark in the header. & N/A \\ \midrule 
\textbf{Content}                       & MC1 & 3 & Most of the information on the website is well balanced and relevant. & N/A \\ \midrule
\multirow{5}{*}{\textbf{Presentation}}
    & MP1 & 3 & Text size is well readable and the constrast is sufficient. & N/A \\ \cmidrule{2-5} 
    & MP2 & 3 & Most label convey their meaning, albeit some link labels are unnecessarily varied (``Discover'', ``Explore'', ``Visit''). It's not always clear when a link points to an external website. & (e.g.) Itinieraries \\ \cmidrule{2-5} 
    & MP3 & 2 & There is unnecessary variety between certain labels: some have an arrow at the end, some are underlined. Labels that mark other topics are consistent (on a white background). & Rents, Arriving and Getting Around\\ \cmidrule{2-5} 
    & MP4 & 2 & Most elements are well positioned, except for the group of links at the top of long pages, which is hard to read and requires the user to scroll back at each to quickly find a topic section. & (e.g.) Universities \\ \cmidrule{2-5} 
    & MP5 & 3 & Group pages have a consistent grid layout. Topic pages are more varied, but the order is usually the same, with a carousel of related topic at the end. Certain section of the website also have a different accent color (e.g. Study is blue, while most of the website is yellow). & Study
\end{tabularx}

\pagebreak

\subsection{Evaluator: Carlo}
\begin{tabularx}{\linewidth}{c c X p{3cm}}
    % \toprule
    \textbf{Heuristic} & \textbf{Score} & \textbf{Comment} & \textbf{Page Url}
    \\ \midrule
    \endfirsthead
    \toprule
    \textbf{Heuristic} & \textbf{Score} & \textbf{Comment} & \textbf{Page Url}
    \\ \midrule
    \endhead
    \midrule
    \footnotesize [Continues on next page]
    \endfoot
    \bottomrule
    \endlastfoot
    % body
    H1 & 1 & Bread crumbs are not consistently used, since in certain pages they are missing. By using the research bar the website sometimes gives no feedback on the status of the process & Travel-info, Itineraries\\ \midrule
    H2 & 2 & The language used is sufficiently clear, still there are some inconsistency in the terms used that are not necessary and might confuse the user. & Shopping-fashion, Rinascente \\ \midrule
    H3 & 3 & No relevant issues were detected. & Restaurants, IH-Hotels\\ \midrule
    H4 & 2 & Standards are proficiently used, nevertheless somethimes they have slightly unespected behaviour (for instance the plus icon that instead of showing more options leads to another page). In other pages index elements looks like links & Rinascente \\ \midrule
    H5 & 3 & Getting to the wrong page is most of the time avoided by effective error prevention & Study\\ \midrule
    H6 & 3 & The website provides sets of options in order to minimize the application of user's memory & Hotels\\ \midrule
    H7 & 2 & Landmarks are present but could be improved in most of the pages by adding few more options. & Come-arrivare, Chiosetto\\ \midrule
    H8 & 2 & Clear example of information overload, still this is an exeption which does not occur frequently & Universities\\ \midrule
    H9 & 1 & By using the research bar the websites sometimes freezes without giving any kind of explanation or feedback on the status of the process & Homepage\\ \midrule
    H10 & n/a & &
\end{tabularx}
    
\begin{tabularx}{\linewidth}{l c c X p{3cm}}
\toprule
\textbf{Category} & \textbf{Heuristic} & \textbf{Score} & \textbf{Comment} & \textbf{Page Url} \\
\midrule
\endfirsthead
\toprule
\textbf{Category} & \textbf{Heuristic} & \textbf{Score} & \textbf{Comment} & \textbf{Page Url} \\
\midrule
\endhead
\midrule
\footnotesize [Continues on next page]
\endfoot
\bottomrule
\endlastfoot

\multirow{5}{*}{\textbf{Navigation}}   
    & MN1 & 3 & The interaction capability is consistent between pages of the same type. Sometimes there is some minor differences but no relevant problem was detected. & Travel-info, Shopping-fashion \\ \cmidrule{2-5} 
    & MN2 & 2 & Navigation between items of the same group is simple and encouraged. Nonetheless sometimes the classification of groups is not evident and that might create confusion & Venues\\ \cmidrule{2-5} 
    & MN3 & 2 & Structural navigation sometimes is facilitated (as in the first link, where sections of the page are displayed with interactable lable on the top of the page) and sometimes it isn't (as in the second link, where an index of the contents is not available). & FAQ covid, events-month\\ \cmidrule{2-5} 
    & MN4 & 2 & Suggestions are provided in order to let the user to switch between similar topics, still the possibility of coming back from a single topic to the previous one is not present & miart\\ \cmidrule{2-5} 
    & MN5 & 1 & Landmarks are present and are sometimes well used as in the second link. In many pages though (such as the page displayed in the first link) landmarks are too little used & adi-design, Restaurants\\ \midrule
\textbf{Content}                       & MC1 & 1 & There are some relevant cases of information overload. The index of the highlighted link is a clear example: for instance it would have been more suitable to group universities by field of studies & Universities  \\ \midrule
\multirow{5}{*}{\textbf{Presentation}} & MP1 & 3 & The text is always readable and of the appropriate size & adi-design \\ \cmidrule{2-5} 
    & MP2 & 2 & The comment here is similar to H4. Still, there are some exeptions (one is the plus icon, as stated above)& Rinascente, Chiosetto\\ \cmidrule{2-5} 
    & MP3 & 1 & Wording is sometimes inconsistent between labels of the same type & Shopping-fashion, Rinascente\\ \cmidrule{2-5} 
    & MP4 & 1 & Relevant and useful feature of the website are almost hidden in the footer. Some of those functionalities are not reachable in other ways. & Itineraries \\ \cmidrule{2-5} 
    & MP5 & 3 & Layout of similar pages is mostly the same across the whole website & adi-design, Castello Sforzesco
\end{tabularx}

\pagebreak

\subsection{Evaluator: Fabio}
\begin{tabularx}{\linewidth}{c c X p{3cm}}
    % \toprule
    \textbf{Heuristic} & \textbf{Score} & \textbf{Comment} & \textbf{Page Url}
    \\ \midrule
    \endfirsthead
    \toprule
    \textbf{Heuristic} & \textbf{Score} & \textbf{Comment} & \textbf{Page Url}
    \\ \midrule
    \endhead
    \midrule
    \footnotesize [Continues on next page]
    \endfoot
    \bottomrule
    \endlastfoot
    % body
    H1 & 1 & The lack of bread crumbs is not a problem in the homepage.
    In all the other links there are mistakes because either the bread crumbs are missing or the bread crumbs include "Home" & All links (except the homepage)\\ \midrule
    H2 & 2 & There are universally recognized icons (e.g. Profile, Search, Menu). But there is the icon of an alarm clock instead of a clock. & Homepage, Restaurants \\ \midrule
    H3 & 0 & When I insert date and number of people and I search, then is not possible anymore to change this data.& Hotel page\\ \midrule
    H4 & 2 & The button "Contacts" looks like a button, but it's not. The same with the button "Map" & FAQ Coronavirus, 2 days Milano\\ \midrule
    H5 & 1 & In the restaurant page there is no error for wrong compilation of the form. Similar problem when registering to the newsletter& Chiosetto, Homepage\\ \midrule
    H6 & 3 & Whenever possible, this heuristic is always satisfied& All links\\ \midrule
    H7 & 2 & The landmarks are useful, except the one about the Profile & All links\\ \midrule
    H8 & 2 & The button about the Social Networks partially cover the text sometimes & Universities, Rents\\ \midrule
    H9 & 1 & In the hotels page, if the geolocalization service fails, the website gives a not helpful error message. Instead in the restaurants page, the same error gives to the user instructions about what to do. & Hotels \\ \midrule
    H10 & n/a & &
\end{tabularx}
    
\begin{tabularx}{\linewidth}{l c c X p{3cm}}
\toprule
\textbf{Category} & \textbf{Heuristic} & \textbf{Score} & \textbf{Comment} & \textbf{Page Url} \\
\midrule
\endfirsthead
\toprule
\textbf{Category} & \textbf{Heuristic} & \textbf{Score} & \textbf{Comment} & \textbf{Page Url} \\
\midrule
\endhead
\midrule
\footnotesize [Continues on next page]
\endfoot
\bottomrule
\endlastfoot

\multirow{5}{*}{\textbf{Navigation}}   
    & MN1 & 3 & & \\ \cmidrule{2-5} 
    & MN2 & 2 & Reaching the Venues page is difficult. & Venues\\ \cmidrule{2-5} 
    & MN3 & 3 & Navigate among Map, Gallery, Contacts is simple & La rinascente\\ \cmidrule{2-5} 
    & MN4 & 3 & There are sections like "what's nearby" and "on the same theme" & Castello Sforzesco\\ \cmidrule{2-5} 
    & MN5 & 2 & Same as H7 &\\ \midrule
\textbf{Content}                       & MC1 & 2 & Information overload is managed well (except in the university page), thanks to "read more" and "see all". & University  \\ \midrule
\multirow{5}{*}{\textbf{Presentation}} & MP1 & 3 & & \\ \cmidrule{2-5} 
    & MP2 & 2 & The text "Fly to Milano" doesn't reflect the goal of the website, that is booking hotels and not flights. The meaning of the label "Top" in the search bar is not clear & Hotels, Venues\\ \cmidrule{2-5} 
    & MP3 & 2 & Some elements have different text but same meaning & Chiosetto\\ \cmidrule{2-5} 
    & MP4 & 3 &  &\\ \cmidrule{2-5} 
    & MP5 & 3 & &
\end{tabularx}

\end{small}